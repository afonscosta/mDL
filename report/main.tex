
\documentclass[11pt]{article}
\usepackage[portuges]{babel}
\usepackage[utf8]{inputenc}
\usepackage{a4wide}
\usepackage{graphicx}
\usepackage[hidelinks]{hyperref}
\usepackage[toc,page]{appendix}
\usepackage{placeins}
\usepackage{float}
\usepackage{indentfirst}
\usepackage{subfiles}
\usepackage{fancyvrb}
\usepackage{soul}
\usepackage{color}
\usepackage[table,xcdraw]{xcolor}
\usepackage{multicol}

% JSON
\usepackage{listings}
\usepackage{xcolor}

\colorlet{punct}{red!60!black}
\definecolor{background}{HTML}{EEEEEE}
\definecolor{delim}{RGB}{20,105,176}
\colorlet{numb}{magenta!60!black}

\lstdefinelanguage{json}{
    basicstyle=\normalfont\ttfamily,
    showstringspaces=false,
    breaklines=true,
    frame=lines,
    backgroundcolor=\color{background},
    string=[s]{"}{"},
    stringstyle=\color{numb},
    comment=[l]{:},
    commentstyle=\color{black},
    %literate=
    % *{0}{{{\color{numb}0}}}{1}
    %  {:}{{{\color{punct}{:}}}}{1}
    %  {,}{{{\color{punct}{,}}}}{1}
    %  {\{}{{{\color{delim}{\{}}}}{1}
    %  {\}}{{{\color{delim}{\}}}}}{1}
    %  {[}{{{\color{delim}{[}}}}{1}
    %  {]}{{{\color{delim}{]}}}}{1},
}
%

\lstset{
    inputencoding=utf8,
    extendedchars=true,
    literate=
        {á}{{\'a}}1 {é}{{\'e}}1 {í}{{\'i}}1 {ó}{{\'o}}1 {ú}{{\'u}}1
        {Á}{{\'A}}1 {É}{{\'E}}1 {Í}{{\'I}}1 {Ó}{{\'O}}1 {Ú}{{\'U}}1
        {à}{{\`a}}1 {è}{{\`e}}1 {ì}{{\`i}}1 {ò}{{\`o}}1 {ù}{{\`u}}1
        {À}{{\`A}}1 {È}{{\'E}}1 {Ì}{{\`I}}1 {Ò}{{\`O}}1 {Ù}{{\`U}}1
        {ä}{{\"a}}1 {ë}{{\"e}}1 {ï}{{\"i}}1 {ö}{{\"o}}1 {ü}{{\"u}}1
        {Ä}{{\"A}}1 {Ë}{{\"E}}1 {Ï}{{\"I}}1 {Ö}{{\"O}}1 {Ü}{{\"U}}1
        {â}{{\^a}}1 {ê}{{\^e}}1 {î}{{\^i}}1 {ô}{{\^o}}1 {û}{{\^u}}1
        {Â}{{\^A}}1 {Ê}{{\^E}}1 {Î}{{\^I}}1 {Ô}{{\^O}}1 {Û}{{\^U}}1
        {ã}{{\~a}}1 {õ}{{\~o}}1
        {Ã}{{\~A}}1 {Õ}{{\~O}}1
        {œ}{{\oe}}1 {Œ}{{\OE}}1 {æ}{{\ae}}1 {Æ}{{\AE}}1 {ß}{{\ss}}1
        {ç}{{\c c}}1 {Ç}{{\c C}}1 {ø}{{\o}}1 {å}{{\r a}}1 {Å}{{\r A}}1
        {€}{{\EUR}}1 {£}{{\pounds}}1,
}

\graphicspath{{images/}{../images/}}

\renewcommand\appendixtocname{Anexos}
\renewcommand\appendixpagename{Anexos}

\begin{document}
\sloppy

\begin{titlepage}

% Defines a new command for the horizontal lines, change thickness here
\newcommand{\HRule}{\rule{\linewidth}{0.5mm}}

\center % Center everything on the page
 
% HEADING SECTIONS
\textsc{\LARGE Engenharia de Segurança}\\[1.5cm]
\textsc{\Large Universidade do Minho}\\[0.5cm]
\textsc{\large Mestrado Integrado em Engenharia Informática}\\[0.6cm]

% TITLE SECTION
\vspace{0.8cm}
\HRule \\[0.6cm]
{ \huge \bfseries Projeto em Identificação \textit{mobile}}\\[0.4cm]
{ \Large \bfseries \textbf{mDL} (\textit{mobile Driving License})} \\[0.4cm]
\HRule \\[1.2cm]

% AUTHOR SECTION
\Large \emph{Autores:}\\
A77531 - Daniel Maia\\
A78034 - Diogo Costa\\
A77364 - Mafalda Nunes\\[1.3cm]

% DATE SECTION
{\large \today}\\[1.5cm]

% LOGO SECTION
\includegraphics[width=0.55\textwidth]{images/logo}\\[1cm]

\vfill % Fill the rest of the page with whitespace

\end{titlepage}


\begin{abstract}
O presente trabalho aborda a temática da identificação \textit{mobile}, mais especificamente da \textit{mobile Driving License} (\texttt{mDL}).

Este documento divide-se em duas partes fundamentais, sendo estas a análise do ISO de desmaterialização da Carta de Condução e a implementação da estrutura que dá suporte à \texttt{mDL}, sem perder de vista os algoritmos, as primitivas criptográficas e os \textit{workflows} que garantem a segurança do \texttt{mDL}.
\end{abstract}

\newpage

\tableofcontents

\newpage

\section{Introdução}
\subfile{sections/introducao}

\section{Contextualização}
\subfile{sections/contextualizacao}

\section{Requisitos}
\subfile{sections/requisitos}

\section{Estrutura de Dados Lógica}
\subfile{sections/estrutura_dados}

\section{Transferência de Dados da mDL}
\subfile{sections/trans_dados}

\section{Mecanismos de Proteção de Dados da mDL}
\subfile{sections/mec_seguranca}

\section{Implementação da mDL (ISO \textit{compliant})}
\subfile{sections/implementacao}

\newpage
\section{Conclusões}
\subfile{sections/conclusoes}


\newpage
\bibliographystyle{unsrt}
\bibliography{biblio}

\end{document}
