
Este trabalho permitiu, através da análise do ISO de desmaterialização da carta de condução, perceber quais os principais constituintes da mesma, bem como os processos necessários para garantir a segurança dos dados, quer no seu armazenamento, quer na sua transmissão. Contudo, algumas das componentes do \texttt{mDL} não foram exploradas ao pormenor, nem implementadas, uma vez que eram opcionais ou envolviam a utilização de dispositivos físicos que não entravam no âmbito deste projeto. Para além disso, verificaram-se algumas dificuldades na compreensão de determinadas componentes do ISO, devido à sua especificidade e constante referência a \textit{standards} externos.

De seguida, na fase de implementação, escolheu-se a linguagem \texttt{Python} devido à disponibilidade de várias bibliotecas criptográficas e à facilidade/rapidez de desenvolvimento associada. No entanto, encontraram-se dificuldades na codificação \texttt{ASN.1} exigida pelo ISO, uma vez que as bibliotecas já existentes não respeitavam todos os requisitos impostos. Desta forma, houve o trabalho adicional de desenvolver um \textit{parser} genérico, capaz de converter as informações dos DG's para o formato hexadecimal desejado. Em termos de mecanismos de segurança, teve de se criar certificados, bem como gerar os \textit{digests} de cada DG e a assinatura do conjunto dos mesmos. Para exemplificar a utilização da estrutura de dados desenvolvida (o \texttt{mDL}), animou-se uma transação de dados entre o titular e um verificador, tendo em conta o processo descrito no ISO.

Por fim, salienta-se que se deverá implementar, como trabalho futuro, a incorporação dos conjuntos de dados não considerados ou de \textit{compact encoding} (para comunicação entre dispositivos com reduzidos recursos de armazenamento).