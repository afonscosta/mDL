
Os requisitos funcionais abrangidos por esta parte do \textit{standard} para a
solução do mDL incluem:

\begin{itemize}
	\item Capcidade de funcionar durante verificação num ambiente \textit{offline} (leitor mDL \textit{offline} e mDL \textit{offline}).
	\item Capcidade de funcionar durante verificação num ambiente \textit{online} (leitor mDL \textit{online} e mDL \textit{online}).
	\item Mecanismos ou uma arquitetura que permite a partes
	interessadas na mDL (titular, aplicador da lei ou entidade privada)
	estabelecer confiança na informação providenciada pela mDL, isto é, ter
	garantias de que a mDL foi emitida pela alegada autoridade de emissão e que
	informação não foi alterada.
	\item Confirmar a ligação entre uma mDL e um titular de mDL.
	\item Transmitir privilégios de condução.
	\item Permitir a leitura de informação entre autoridades emissoras.
	\item Permitir que um titular de mDL autorize a libertação de informação especificamente selecionada da mDL para um leitor mDL.
\end{itemize}

Existem ainda requisitos técnicos relativos à interface entre uma mDL e um
leitor mDL, que são especificados nesta parte do ISO/IEC 18013, nomeadamente:

\begin{itemize}
	\item Estrutura de dados lógicos com as informações da mDL, quando transferidas entre uma mDL e um leitor mDL, deve respeitar os seguintes aspetos:
	\begin{itemize}
		\item Elementos de dados considerados no ISO/IEC 18013-2:
		\begin{itemize}
			\item DG1: elementos de texto (obrigatório).
			\item DG2: detalhes do titular da licença (opcional).
			\item DG3: detalhes da autoridade de emissora (opcional).
			\item DG4: imagem do retrato do titular da licença (opcional).
			\item DG5: assinatura / imagem de marca habitual (opcional).
			\item DG6: modelo biométrico facial (opcional).
			\item DG7: modelo biométrica do dedo (opcional).
			\item DG8: modelo biométrico da íris (opcional).
			\item DG9: outro modelo biométrico (opcional).
			\item DG10: reservado para uso futuro.
			\item DG11: dados domésticos (opcional).
		\end{itemize}

		\item Elementos de dados adicionais:
		\begin{itemize}
			\item Inclusão obrigatória da imagem facial do titular.
			\item Elementos de dados adicionais para ``Up to Date info''.
			\item \hl{Identificador adicional que indica o fator de forma.}
			\item Grupos de dados mDL, utilizados para transferência de informação seletiva (inclui novos elementos de dados).
		\end{itemize}
	\end{itemize}

	\item Protocolo de comunicação para troca de dados mDL, entre uma mDL e um leitor:
	\begin{itemize}
		\item Camada de transmissão:
		\begin{itemize}
			\item ISO/IEC 14443 e/ou ISO/IEC 18092 (NFC).
			\item Interface visual (câmara).
			\item Wi-Fi \textit{Aware}.
			\item Internet.
			\item \textit{Bluetooth Low Energy} (BLE).
		\end{itemize}
		\item Camada de apresentação:
		\begin{itemize}
			\item Comandos ISO/IEC 7816-4 e ISO/IEC 7816-8 (Parte 2 e 3) para o equivalente a \textit{Standard Encoding} para mDL.
			\item Códigos de barras 2D (para estabelecimento de conexão entre dispositivos e o equivalente a \textit{Compact Encoding} para mDL, na transferência de dados da mDL).
		\end{itemize}
	\end{itemize}

	\item Mecanismos de proteção de dados para serem aplicados, tendo em conta o ISO/IEC 18013-3 - preservar confidencialidade, integridade e autenticação de dados mDL.
\end{itemize}

Especificam-se ainda alguns requisitos funcionais relativos a uma aplicação de leitores de mDL, para assegurar a verificação fiável de uma mDL:

\begin{itemize}
	\item Disponibilidade de verificação de dados (e.g. certificados digitais) de autoridades emissoras, incluindo a definição do modelo de confiança utilizado para uma mDL.
	\item Sequência de leitura para dados de uma mDL.
	\item Sequência de verificação para dados de uma mDL.
\end{itemize}

Assim sendo, todos estes requisitos são necessários para assegurar a competência do sistema.
