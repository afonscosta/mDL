% - [ ] DG's (1, 6, 10) + COM + GroupAccess + SOD [DANIEL]
%     - [ ] Variáveis de instância
%     - [ ] Funcionalidades (API)
%     - [ ] JSON específico

% Excecionalmente, o ficheiro \texttt{EF\_GroupAccess} é responsável pelo estabelecimento e gestão de permissões de acesso dos grupos de dados. Por fim, o ficheiro \texttt{EF\_SOD} é responsável por guardar a \textit{hash} coletiva do sistema, bem como armazenar os seus certificados e assinaturas digitais.

% Para facilitar o acesso aos dados, foi desenvolvido também uma estrutura JSON para cada um deles que delimita a ordem de entrada e tipo de cada um dos campos para o respetivo DG. Para cada campo, é definida a sua \textit{tag} (se relevante), o seu comprimento em \textit{bytes} e quaisquer limitações de \textit{input} a qual esteja sujeito.

Para cada um dos ficheiros que compõe a estrutura de dados, foi implementada uma \textit{script} com um conjunto de funcionalidades que as permitem criar e gerir os respetivos ficheiros. Estes são depois acedidos pelo \textit{mDL} através da sua API, da qual as \textit{scripts} partilham a capacidade de carregar e escrever os dados do respetivo grupo de dados em ficheiro:

\begin{lstlisting}[language=Python, caption=Exemplos de funções de inicialização\, escrita e carregamento de dados no DG1]
    def __init__(self, data):
    """
    Parâmetros
    ----------
    data : dict
        dicionário com os nomes das variáveis de instância como chaves
        e respetivo conteúdo como valor
    """
    if isinstance(data, str):
        data = self.load(data)
    self.family_name = data['family_name']
    self.name = data['name']
    self.date_of_birth = data['date_of_birth']
    self.date_of_issue = data['date_of_issue']
    self.date_of_expiry = data['date_of_expiry']
    self.issuing_country = data['issuing_country']
    self.issuing_authority = data['issuing_authority']
    self.license_number = data['license_number']
    self.number_of_entries = data['number_of_entries']
    self.categories_of_vehicles = data['categories_of_vehicles']

    def save(self, filename):
        """Armazena os dados do DG1 num ficheiro, codificados com ASN1.

        Parâmetros:
        -----------
        filename : str
            nome do ficheiro onde se pretende armazenar os dados
        """
        with open(filename, 'w+') as fp:
            hex_data = asn1.encode(self, './data_groups/configs/dg1.json')
            fp.write(hex_data)

    def load(self, filename):
        """ Carrega os dados do DG1 de um ficheiro, codificado com ASN1.

        Parâmetros
        ----------
        filename : str
            nome do ficheiro a partir do qual se pretende obter os dados

        Retorna
        -------
        data : dict
            dicionário com os nomes das variáveis de instância como chaves
            e respetivo conteúdo como valor
        """
        with open(filename, 'r') as fp:
            data = asn1.decode(fp.read(), './data_groups/configs/dg1.json')
        return data
\end{lstlisting}

Tanto o carregamento como a escrita destes é executada com a ajuda de um ficheiro JSON que permite alterações à estrutura da implementação sem que haja alterações no código em si. Cada um dos DG's tem o seu próprio JSON no qual são guardadas os seus campos, bem como os respetivos comprimentos em \textit{bytes} e as suas \textit{tags} e é convertido de e para o formato ASN1 consoante a sua leitura e escrita, respetivamente.

\subsection{Data Groups 1, 6 e 10}

Estes DG's são responsáveis por armazenar os dados do titular mDL, bem como os do próprio título mDL. Estes contêm um conjunto próprio de funções que as permitem gerar dicionários de dados com a respetiva informação mDL bem como métodos dedicados à sua codificação no formato ASN1 e o seu \textit{hashing} de acordo com um algoritmo da sua escolha:

\begin{lstlisting}[caption=API comum aos DG's exemplificada pelo DG1, language=Python]
    def encode(self):
        """Codificados os dados do DG1 com ASN1.

        Retorna:
        --------
        hex_data : str
            dados do DG1 codificados em hexadecimal
        """
        return asn1.encode(self, './data_groups/configs/dg1.json')

    def __str__(self):
    """ Retorna uma string que apresenta os dados do DG1

    Retorna
    -------
    data : str
        string com os dados do DG1
    """
    return ';'.join([self.family_name,\
            self.name,\
            self.date_of_birth,\
            self.date_of_issue,\
            self.date_of_expiry,\
            self.issuing_country,\
            self.issuing_authority,\
            self.license_number,\
            str(self.number_of_entries),\
            str(self.categories_of_vehicles)])

            def get_data(self):
            """ Devolve os dados associados.
    
            Retorna
            -------
            data : dict
                dicionário com os nomes das variáveis de instância como
                chaves e respetivo conteúdo como valor
            """
            data = {}
            data['family_name'] = self.family_name
            data['name'] = self.name
            data['date_of_birth'] = self.date_of_birth
            data['date_of_issue'] = self.date_of_issue
            data['date_of_expiry'] = self.date_of_expiry
            data['issuing_country'] = self.issuing_country
            data['issuing_authority'] = self.issuing_authority
            data['license_number'] = self.license_number
            data['number_of_entries'] = self.number_of_entries
            data['categories_of_vehicles'] = self.categories_of_vehicles
            return data
    
        def hash(self, oid):
            """ Calcula o valor de hash dos dados do DG1.
    
            Retorna
            -------
            digest : str
                valor de hash
            """
            data = self.family_name + self.name +\
                self.date_of_birth + self.date_of_issue +\
                self.date_of_expiry + self.issuing_country +\
                self.issuing_authority + self.license_number +\
                "".join(self.categories_of_vehicles) +\
                str(self.number_of_entries)
            if oid == 'id-sha1':
                digest = hashes.Hash(hashes.SHA1(), backend=default_backend())
                digest.update(data.encode())
                return digest.finalize()
            elif oid == 'id-sha224':
                digest = hashes.Hash(hashes.SHA224(), backend=default_backend())
                digest.update(data.encode())
                return digest.finalize()
            elif oid == 'id-sha256':
                digest = hashes.Hash(hashes.SHA256(), backend=default_backend())
                digest.update(data.encode())
                return digest.finalize()
            elif oid == 'id-sha384':
                digest = hashes.Hash(hashes.SHA384(), backend=default_backend())
                digest.update(data.encode())
                return digest.finalize()
            elif oid == 'id-sha512':
                digest = hashes.Hash(hashes.SHA512(), backend=default_backend())
                digest.update(data.encode())
                return digest.finalize()
            else:
                print('ERROR: Hash algorithm not implemented.')
                sys.exit(1)
\end{lstlisting}

\subsubsection{Data Group 1}

Este DG é responsável por armazenar os dados do titular mDL, bem como alguma informação básica do próprio título mDL. Cada um destes contém um conjunto de variáveis de instância correspondentes aos respetivos dados explicitados na estrutura de dados:

\begin{lstlisting}[caption=Instanciação das variáveis do DG1, language=Python]
    self.family_name = data['family_name']
    self.name = data['name']
    self.date_of_birth = data['date_of_birth']
    self.date_of_issue = data['date_of_issue']
    self.date_of_expiry = data['date_of_expiry']
    self.issuing_country = data['issuing_country']
    self.issuing_authority = data['issuing_authority']
    self.license_number = data['license_number']
    self.number_of_entries = data['number_of_entries']
    self.categories_of_vehicles = data['categories_of_vehicles']
\end{lstlisting}

\subsubsection{Data Group 6}

Este DG contém um conjunto de dados biométricos relativos ao titular, em particular, a sua foto. Para permitir uma melhor gestão deste ficheiro, foi criada uma classe auxiliar \texttt{BiometricTemplate} que guarda cada um dos dados biométricos do DG:

\begin{lstlisting}[caption=Instanciação dos dados do DG6, recorrendo à classe auxiliar, language=Python]
class BiometricTemplate:
    def __init__(self, version, bdb_owner, bdb_type, bdb):
        self.version = version if version is not None else '0101'
        self.bdb_owner = bdb_owner
        self.bdb_type = bdb_type
        self.bdb = bdb
    
    def __str__(self):
        return '(' + ', '.join([str(self.version), str(self.bdb_owner), str(self.bdb_type), str(self.bdb)[:20] + '...']) + ')'


class DG6:
    """
    CLasse utilizada para representar um grupo de dados 6.
    
    Atributos
    ---------
    
    Métodos
    -------
    """
    def __init__(self, data):
        """ Construtor da classe DG6.

        Parâmetros
        ----------
        data : list
            lista de dicionários que representam biometric templates
        """
        if isinstance(data, str):
            data = self.load(data)
        self.biometric_templates = []
        
        for template in data['biometric_templates']:
            self.biometric_templates.append(
                BiometricTemplate(
                    template['version'],
                    template['bdb_owner'],
                    template['bdb_type'],
                    template['bdb']
                )
            )
        self.number_of_entries = data['number_of_entries']
\end{lstlisting}


\subsubsection{Data Group 10}

Este grupo de dados é responsável por armazenar os dados relativos à gestão do mDL, incluindo a sua data de validade e as datas do último e próximo \textit{updates}.

\begin{lstlisting}[caption=Instanciação dos dados do DG10, language=Python]
    self.version = data['version']
    self.last_update = data['last_update']
    self.expiration_date = data['expiration_date']
    self.next_update = data['next_update']
    self.management_info = data['management_info']
\end{lstlisting}


\subsection{EF_COM}

O ficheiro elementar COM é responsável por armazenar as \textit{tags} relativas aos grupos de dados presentes numa dada implementação do mDL.

\begin{lstlisting}[caption=Instanciação de variáveis no EF.COM, language=Python]
    self.version = data['version']
    self.tag_list = data['tag_list']
\end{lstlisting}

Como esta funciona puramente como um meio rápido de averiguar os dados disponíveis ao mDL, este contém apenas as funções de API básica, bem como uma função de exibição de dados semelhante aos grupos de dados anteriormente expostos.


\subsection{EF_GroupAccess}

