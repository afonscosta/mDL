
Este projeto é desenvolvido no âmbito da unidade curricular de Engenharia de Segurança, do Mestrado Integrado em Engenharia Informática, da Universidade do Minho.

Um dos principais objetivos deste trabalho é a investigação e análise do \textit{standard} ISO de desmaterialização da Carta de Condução, mais especificamente o ``\texttt{ISO/IEC CD 18013-5 Information technology} -- \texttt{Personal identification} -- \texttt{ISO compliant driving licence} -- \texttt{Part 5: Mobile driving licence application (mDL)}''. Pretende-se dar especial atenção à estrutura de dados requerida e aos vários algoritmos, primitivas criptográficas e \textit{workflows} que garantem a segurança da mDL. Por fim, deverá apresentar-se uma implementação da mDL, de acordo com o ISO, através da utilização de bibliotecas \textit{open-source}.

De facto, esta desmaterialização de documentos, que se baseia em técnicas e algoritmos criptográficos, torna possível o acesso aos mesmos através de dispositivos móveis, que são comummente utilizados na atualidade. Assim, começa a surgir a tendência de substituir os documentos de identificação, como hoje os conhecemos (em papel ou \textit{smartcard}), por documentos desmaterializados.
