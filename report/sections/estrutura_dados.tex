A estrutura de dados mDL é codificada como um conjunto de objetos de dados BER-TLV e pode ser apresentada em dois formatos: \textit{standard encoding} e \textit{compact encoding}.

% TODO: Explicar BER-TLV

\subsection{\textit{Standard encoding}}

%O \textit{standard encoding} do mDL é constituído por três componentes primários: A estrutura de ficheiros, o conjunto de comandos e os grupos de dados que o constituem.

\subsubsection{Estrutura de ficheiros}

A estrutura de dados lógica do mDL é constituída por um conjunto de ficheiros elementares, cada um deles contendo um ou mais grupos de dados. Cada um destes pode ser classificado como obrigatório, opcional ou condicional (dependendo do suporte providenciado) na implementação do mDL. Relativamente à permissão de acesso a um dado ficheiro, é necessário indicar se o consentimento explícito é requerido do titular mDL.

Na tabela \ref{table:ficheiros_elementares} são identificados os vários conjuntos de ficheiros elementares que constituem a estrutura de dados lógica do mDL.

\begin{table}[H]
\centering
\caption{Conjunto de ficheiros elementares.}
\label{table:ficheiros_elementares}
\vspace{0.2cm}
\begin{tabular}{|l|l|l|}
\hline
\rowcolor[HTML]{EFEFEF}
\textbf{Ficheiro Elementar} & \textbf{Presença} & \textbf{Consentimento} \\ \hline
Data group 1 & Obrigatória & Explícito \\ \hline
Data group 2-4 & Opcional & Explícito \\ \hline
Data group 5 & \begin{tabular}[c]{@{}l@{}}Opcional\\ (Não recomendado)\end{tabular} & Explícito \\ \hline
Data group 6 & Obrigatória &  \\ \hline
Data group 7-9 & Opcional & Explícito \\ \hline
Data group 10 & Obrigatória &  \\ \hline
Data group 11 & Opcional & Explícito \\ \hline
Data group 13 & \begin{tabular}[c]{@{}l@{}}Condicional\\ (Obrigatória se Active \\ Authentication é suportada)\end{tabular} &  \\ \hline
Data group 14 & \begin{tabular}[c]{@{}l@{}}Condicional\\ (Obrigatória se autenticação\\ PACE e/ou Chip é suportada)\end{tabular} &  \\ \hline
Data group 32-127 & Opcional & Explícito \\ \hline
EF.COM & Obrigatória &  \\ \hline
EF.SOD & Obrigatória &  \\ \hline
EF.CardAccess & \begin{tabular}[c]{@{}l@{}}Condicional\\ (Obrigatória se PACE é \\ suportada)\end{tabular} &  \\ \hline
EF.GroupAccess & Obrigatória &  \\ \hline
\end{tabular}
\end{table}


%\begin{\begin{itemize}
%    \item A estrutura
%    \item
%    \item 
%\end{itemize}

\subsubsection{Comandos}

Os comandos de uma aplicação mDL cumprem a norma ISO/IEC 18013-2. Cada comando toma a forma de uma mensagem que será transmitida a um recipiente, sendo constituída por um cabeçalho e um corpo.

O cabeçalho é constituído por quatro \textit{bytes}, cada um dos quais indicando um campo, presentes na seguinte ordem:
\begin{itemize}
    \item O \textit{byte Class} (CLA) que, como o nome indica, especifica a classe, interindústria ou proprietária, do comando a executar. Indica também, caso se trate de um comando de classe interindústria, se se pretende executar \textit{chaining} de comandos e respostas (e.g. transmissão de uma \textit{string} demasiado longa para um único comando). Adicionalmente, indica se se pretende utilizar um canal seguro para a transmissão de dados e o respetivo formato. Por fim, é indicado o canal lógico sobre o qual a transmissão será efetuada.

    \item O \textit{byte Instruction} (INS), que especifica exatamente qual comando será processado. Existe uma variedade de comandos providenciados pela norma ISO/IEC 18013-2. Para além dos comandos detalhados na norma, é especificado um comando adicional \textit{UPDATE BINARY}, que atualiza o ficheiro EF.GroupAccess.%Há 40 comandos na norma mais o UPDATE BINARY, põe-se nos anexos?

    \item Os \textit{bytes Parameter} 1 e 2 (P1 e P2), que indicam controlos e opções para o processamento do comando.
\end{itemize}


\subsubsection{Grupos de dados}

Os dados mDL são organizado em 11 grupos de dados, de acordo com a norma ISO/IEC 18013-2, com algumas alterações. 

O primeiro grupo de dados (DG 1) é responsável por guardar o conjunto mínimo de dados essenciais para identificação internacional, com a exceção da assinatura e foto do indivíduo. Os grupos de dados 6 e 10, tornam-se obrigatórios na implementação do mDL. 

\begin{table}[H]
\centering
\caption{Grupos de dados do mDL.}
\label{table:grupos_dados}
\vspace{0.2cm}
\begin{tabular}{|l|l|}
\hline
\rowcolor[HTML]{EFEFEF}
Ficheiro Elementar & Conteúdo \\ \hline
Data group 1 & Elementos obrigatórios \\ \hline
Data group 2 & Detalhes do titular \\ \hline
Data group 3 & Detalhes da autoridade emissora \\ \hline
Data group 4 & Foto do titular \\ \hline
Data group 5 & Assinatura \\ \hline
Data group 6 & Biométrica da face \\ \hline
Data group 7 & Biométrica do dedo \\ \hline
Data group 8 & Biométrica da íris \\ \hline
Data group 9 & Outras biométricas \\ \hline
Data group 10 & Dados mDL obrigatórios \\ \hline
Data group 11 & Dados domésticos \\ \hline
\end{tabular}
\end{table}

Para além destes, são introduzidos os grupos de dados opcionais 32 a 127, que permitirão a autorização seletiva de informação mDL para ser fornecida ao leitor. Quaisquer destes grupos que contenha dados é introduzido no elemento EF.SOD. É incluído também o elemento EF.GroupAccess, que contém informação sobre que grupos de dados são disponibilizados ao leitor mDL.

\subsection{\textit{Compact encoding}}

O \textit{Compact encoding} é o esquema de dados utilizado na transferência de informação por meio de uma interface ótica, tais como códigos de barras ou fitas magnéticas. Estes requerem um espaço de armazenamento entre 300 B e 5 kB. Devido a esta limitação, o número de grupos de dados é restrito, bem como o espaço permitido para cada um.

O esquema do \textit{compact encoding} providencia espaço para os grupos de dados 1, 6 e 10 obrigatoriamente, bem como a possibilidade da utilização dos grupos 2, 3, 4, 7 e 11, caso necessário.
