A estrutura de dados mDL é codificada como um conjunto de objetos de dados BER-TLV. 

As \textit{Basic Encoding Rules} (BER) especificam o formato da estrutura de dados, recorrendo a um método de \textit{enconding type-length-value} (TLV). A codificação de dados consiste em quatro elementos que se apresentam na seguinte ordem:

\begin{enumerate}
    \item O octeto \textit{Type}, que distingue os dados de outros membros.
    \item O octeto \textit{Length}, que define o número de bytes que constituem o objeto.
    \item O octeto \textit{Value}, que contém o conteúdo do elemento de dados.
\end{enumerate}

Deste modo, a estrutura mDL pode ser apresentada em dois formatos: \textit{standard encoding} e \textit{compact encoding}.


\subsection{\textit{Standard encoding}}

O \textit{standard encoding} do mDL é constituído por três componentes primários: A estrutura de ficheiros, o conjunto de comandos e os grupos de dados que o constituem.

\subsubsection{Estrutura de ficheiros}

A estrutura de dados lógica do mDL é constituída por um conjunto de ficheiros elementares, cada um deles contendo um ou mais grupos de dados. Cada um destes pode ser classificado como obrigatório, opcional ou condicional (dependendo do suporte providenciado) na implementação do mDL. Relativamente à permissão de acesso a um dado ficheiro, é necessário indicar se o consentimento explícito é requerido do titular mDL.

Na tabela \ref{table:ficheiros_elementares} são identificados os vários conjuntos de ficheiros elementares que constituem a estrutura de dados lógica do mDL.

\begin{table}[H]
\centering
\caption{Conjunto de ficheiros elementares.}
\label{table:ficheiros_elementares}
\vspace{0.2cm}
\begin{tabular}{|l|l|l|}
\hline
\rowcolor[HTML]{EFEFEF}
\textbf{Ficheiro Elementar} & \textbf{Presença} & \textbf{Consentimento} \\ \hline
Data group 1 & Obrigatória & Explícito \\ \hline
Data group 2-4 & Opcional & Explícito \\ \hline
Data group 5 & \begin{tabular}[c]{@{}l@{}}Opcional\\ (Não recomendado)\end{tabular} & Explícito \\ \hline
Data group 6 & Obrigatória &  \\ \hline
Data group 7-9 & Opcional & Explícito \\ \hline
Data group 10 & Obrigatória &  \\ \hline
Data group 11 & Opcional & Explícito \\ \hline
Data group 13 & \begin{tabular}[c]{@{}l@{}}Condicional\\ (Obrigatória se Active \\ Authentication é suportada)\end{tabular} &  \\ \hline
Data group 14 & \begin{tabular}[c]{@{}l@{}}Condicional\\ (Obrigatória se autenticação\\ PACE e/ou Chip é suportada)\end{tabular} &  \\ \hline
Data group 32-127 & Opcional & Explícito \\ \hline
EF.COM & Obrigatória &  \\ \hline
EF.SOD & Obrigatória &  \\ \hline
EF.CardAccess & \begin{tabular}[c]{@{}l@{}}Condicional\\ (Obrigatória se PACE é \\ suportada)\end{tabular} &  \\ \hline
EF.GroupAccess & Obrigatória &  \\ \hline
\end{tabular}
\end{table}

\subsubsection{Comandos}

Os comandos de uma aplicação mDL cumprem a norma ISO/IEC 18013-2. Cada comando toma a forma de uma mensagem que será transmitida a um recipiente, sendo constituída por um cabeçalho e um corpo.

O cabeçalho é constituído por quatro \textit{bytes}, cada um dos quais indicando um campo, presentes na seguinte ordem:
\begin{itemize}
    \item O \textit{byte Class} (CLA) que, como o nome indica, especifica a classe, interindústria ou proprietária, do comando a executar. Indica também, caso se trate de um comando de classe interindústria, se se pretende executar \textit{chaining} de comandos e respostas (e.g. transmissão de uma \textit{string} demasiado longa para um único comando). Adicionalmente, indica se se pretende utilizar um canal seguro para a transmissão de dados e o respetivo formato. Por fim, é indicado o canal lógico sobre o qual a transmissão será efetuada.

    \item O \textit{byte Instruction} (INS), que especifica exatamente qual comando será processado. Existe uma variedade de comandos providenciados pela norma ISO/IEC 18013-2. Para além dos comandos detalhados na norma, é especificado um comando adicional \textit{UPDATE BINARY}, que atualiza o ficheiro EF.GroupAccess.%Há 40 comandos na norma mais o UPDATE BINARY, põe-se nos anexos?

    \item Os \textit{bytes Parameter} 1 e 2 (P1 e P2), que indicam controlos e opções para o processamento do comando.
\end{itemize}


\subsubsection{Grupos de dados}

Os dados mDL são organizado em 11 grupos de dados, de acordo com a norma ISO/IEC 18013-2, com algumas alterações. 

O primeiro grupo de dados (DG 1) é responsável por guardar o conjunto mínimo de dados essenciais para identificação internacional, com a exceção da assinatura e foto do indivíduo. Os grupos de dados 6 e 10, tornam-se obrigatórios na implementação do mDL. 

\begin{table}[H]
\centering
\caption{Grupos de dados do mDL.}
\label{table:grupos_dados}
\vspace{0.2cm}
\begin{tabular}{|l|l|}
\hline
\rowcolor[HTML]{EFEFEF}
Ficheiro Elementar & Conteúdo \\ \hline
Data group 1 & Elementos obrigatórios \\ \hline
Data group 2 & Detalhes do titular \\ \hline
Data group 3 & Detalhes da autoridade emissora \\ \hline
Data group 4 & Foto do titular \\ \hline
Data group 5 & Assinatura \\ \hline
Data group 6 & Biométrica da face \\ \hline
Data group 7 & Biométrica do dedo \\ \hline
Data group 8 & Biométrica da íris \\ \hline
Data group 9 & Outras biométricas \\ \hline
Data group 10 & Dados mDL obrigatórios \\ \hline
Data group 11 & Dados domésticos \\ \hline
\end{tabular}
\end{table}

Para além destes, são introduzidos os grupos de dados opcionais 32 a 127, que permitirão a autorização seletiva de informação mDL para ser fornecida ao leitor. Quaisquer destes grupos que contenha dados é introduzido no elemento EF.SOD. É incluído também o elemento EF.GroupAccess, que contém informação sobre que grupos de dados são disponibilizados ao leitor mDL.

Por motivos de simplicidade, e tendo em conta que se pretende implementar apenas os campos obrigatórios do mDL, serão apenas aprofundadas as constituições dos elementos DG1, DG6, DG10, EF.COM, EF.SOD e EF.GroupAcess.

\subsubsection{Data group 1}

Este ficheiro elementar contém os dados demográficos do titular, bem como as categorias de veículo para os quais este tem qualificações e é identificado pela \textit{tag} \texttt{'61'} e o identificador \texttt{'01'}, seguido pelo comprimento total desto grupo. Como a totalidade da informação contida neste é obrigatória e encontra-se numa ordem fixa, a informação demográfica encontra-se concatenada num único objeto com a \textit{tag} \texttt{'5F1F'} seguida do comprimento da mesma. 

De tal modo, sabendo que cada campo de tamanho variável é precedido pelo respetivo valor de comprimento, os dados demográficos seguem a seguinte estrutura:

\begin{table}[H]
\centering
\caption{Dados armazenados no DG1.}
\label{table:grupos_dados_1}
\vspace{0.2cm}
\begin{tabular}{|c|c|c|c|}
\hline
\rowcolor[HTML]{EFEFEF} 
Nome & \begin{tabular}[c]{@{}c@{}}Tamanho\\Variável/\\Fixo\end{tabular} & Formato & Exemplo \\ \hline
Nome Próprio & Variável & \begin{tabular}[c]{@{}c@{}}Até 36 letras\\ e/ou símbolos\end{tabular} & Smithe-Williams \\ \hline
Apelidos & Variável & \begin{tabular}[c]{@{}c@{}}Até 36 letras\\ e/ou símbolos\end{tabular} & Alexander George Thomas \\ \hline
\begin{tabular}[c]{@{}c@{}}Data de nascimento\\ (yyyymmdd)\end{tabular} & Fixo & 8 números & 19700301 \\ \hline
\begin{tabular}[c]{@{}c@{}}Data de emissão\\ (yyyymmdd)\end{tabular} & Fixo & 8 números & 20020915 \\ \hline
\begin{tabular}[c]{@{}c@{}}Data de expiração\\ (yyyymmdd)\end{tabular} & Fixo & 8 números & 20070930 \\ \hline
País emissor & Fixo & 3 letras & JPN \\ \hline
Autoridade emissora & Variável & \begin{tabular}[c]{@{}c@{}}Até 65 letras, \\números e/ou \\símbolos\end{tabular} & \begin{tabular}[c]{@{}c@{}}HOKKAIDO PREFECTURAL \\ POLICE ASAHIKAWA AREA\\ PUBLIC SAFETY COMMISSION\end{tabular} \\ \hline
Número de licença & Variável & \begin{tabular}[c]{@{}c@{}}Até 65 letras\\ e/ou números\end{tabular} & A290654395164273X \\ \hline
\end{tabular}
\end{table}

Por sua vez, cada uma das qualificações de veículo é codificada com a \textit{tag} \texttt{'7F63'}, seguida pelo comprimento total da lista e pelo número de entradas da mesma, codificado com a \textit{tag} \texttt{'02'}. Cada elemento da lista é codificado com a \textit{tag} \texttt{'87'} seguida pelo comprimento do elemento.



\subsubsection{Data group 6}

Este grupo é responsável pelo armazenamento de uma variedade de identificadores biométricos do titular. No entanto, por motivos de simplicidade, assume-se que apenas a foto estará presente. Este ficheiro é identificado pela \textit{tag} \texttt{'75'} e o identificador \texttt{'06'}, seguido pelo comprimento total do grupo. A seguir segue a seguinte formatação:

\begin{table}[H]
\centering
\caption{Dados armazenados no DG6.}
\label{table:grupos_dados_6}
\vspace{0.2cm}
\begin{tabular}{|c|c|c|}
\hline
\rowcolor[HTML]{EFEFEF}
Tag & Comprimento & Valor \\ \hline
'7461' & Variável & Informação dos subgrupos biométricos \\ \hline
'02' & Variável & \begin{tabular}[c]{@{}c@{}}Número de subgrupos biométricos do grupo\\ \\ (no contexto do Projeto, 1)\end{tabular} \\ \hline
'7F60' & Variável & Subgrupo biométrico \\ \hline
'A1' & Variável & Header do subgrupo biométrico (BHT) \\ \hline
'80' & 02 & Versão da header \\ \hline
'86' & 02 & \begin{tabular}[c]{@{}c@{}}BDB product owner, product type\\ (2 números positivos concatenados)\end{tabular} \\ \hline
'87' & 02 & \begin{tabular}[c]{@{}c@{}}BDB format owner\\ (1 número positivo)\end{tabular} \\ \hline
'88' & 02 & \begin{tabular}[c]{@{}c@{}}BDB format type\\ (1 número positivo)\end{tabular} \\ \hline
'5F2E' & Variável & \begin{tabular}[c]{@{}c@{}}Bloco de dados biométrico\\ (formato definido pelos dois anteriores)\end{tabular} \\ \hline
\end{tabular}
\end{table}



\subsubsection{Data group 10}

Este grupo é responsável por guardar os dados obrigatórios mDL, sendo identificado pela \textit{tag} \texttt{'62'} e o identificador \texttt{'0A'}, seguido pelo comprimento total do grupo. Seguem-se então os seguintes parâmetros:

\begin{table}[H]
\centering
\caption{Dados armazenados no DG10.}
\label{table:grupos_dados_10}
\vspace{0.2cm}
\begin{tabular}{|c|c|c|}
\hline
\rowcolor[HTML]{EFEFEF} 
Tag & Comprimento & Valor \\ \hline
02 & 01 & Versão do mDL \\ \hline
'5F28' & 07 & \begin{tabular}[c]{@{}c@{}}Data do último update. Timestamp em UTC,\\  no formato YYYYMMDDhhmmss\end{tabular} \\ \hline
'5F2B' & 07 & \begin{tabular}[c]{@{}c@{}}Data de expiração do mDL. Timestamp em UTC,\\  no formato YYYYMMDDhhmmss\end{tabular} \\ \hline
'5F38' & 07 & \begin{tabular}[c]{@{}c@{}}Data do próximo update. Timestamp em UTC,\\  no formato YYYYMMDDhhmmss\end{tabular} \\ \hline
'5F39' & Variável & \begin{tabular}[c]{@{}c@{}}Informação de gestão mDL para autoridades,\\ string de octetos\end{tabular} \\ \hline
\end{tabular}
\end{table}


\subsubsection{EF.COM}

Este grupo de dados é responsável pelo armazenamento da versão da estrutura de dados lógica e a lista de \textit{tags} correspondentes aos grupos de dados presentes na implementação do mDL. O EF.COM é identificado pela \textit{tag} \texttt{'60'} e o identificador \texttt{'1E'}, seguido pelo comprimento total do grupo. Seguem-se então os seguintes parâmetros:

\begin{table}[H]
\centering
\caption{Dados armazenados no EF.com}
\label{table:grupos_dados_ef_com}
\vspace{0.2cm}
\centering
\resizebox{\textwidth}{!}{%
\begin{tabular}{|c|c|c|}
\hline
\rowcolor[HTML]{EFEFEF} 
Tag & Comprimento & Valor \\ \hline
'5F01' & Variável & \begin{tabular}[c]{@{}c@{}}Número da versão em formato 'aabb', no qual 'aa' define o maior nível \\ de revisão e 'bb' define o nível de lançamento. Ambos são numéricos\\ e codificados como 2 bytes em formato BCD.\end{tabular} \\ \hline
'5C' & Variável & A lista de tags de grupos de dados. \\ \hline
\end{tabular}%
}\end{table}

\subsubsection{EF.SOD}

O elemento EF.SOD é um documento de segurança identificado pela \textit{tag} \texttt{'77'} e o identificador \texttt{'1D'}. Trata-se de uma assinatura digital gerada através da concatenação das assinaturas de todos os restantes elementos presentes, em formato DER. 

\subsubsection{EF.GroupAccess}

O grupo EF.GroupAccess contem a informação de quais dados estão disponíveis a um leitor mDL, sendo identificado pelo identificador \texttt{'18'}. Este contém apenas a \textit{tag} \texttt{'04'}, o comprimento dos dados e uma \textit{string} de octetos correspondentes aos grupos de dados a que o leitor tem acesso, bem como as respetivas \textit{tags}. Como este elemento é dinâmico, a sua \textit{hash} não está presente no ficheiro EF.SOD.

\subsection{\textit{Compact encoding}}

O \textit{Compact encoding} é o esquema de dados utilizado na transferência de informação por meio de uma interface ótica, tais como códigos de barras ou fitas magnéticas. Estes requerem um espaço de armazenamento entre 300 B e 5 kB. Devido a esta limitação, o número de grupos de dados é restrito, bem como o espaço permitido para cada um.

O esquema do \textit{compact encoding} providencia espaço para os grupos de dados 1, 6 e 10 obrigatoriamente, bem como a possibilidade da utilização dos grupos 2, 3, 4, 7 e 11, caso necessário.
