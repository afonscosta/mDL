
O \textit{standard} ISO/IEC (\textit{International Organization for Standardization} / \textit{International Electrotechnical Commission}) 18013 é caracterizado pelo titulo geral \texttt{Personal Identification — ISO
Compliant Driving Licence} e consiste nas seguintes partes:

\begin{itemize}
	\item \textbf{Parte 1} -- \texttt{Physical Characteristics and Basic Data Set}: descreve as características físicas, o conjunto de elementos de dados básico, o \textit{layout} visual e capacidades de segurança físicas (recursos legíveis pelo ser humano) de uma \textit{ISO-compliant driving licence} (IDL);
	\item \textbf{Parte 2} -- \texttt{Machine-Readable Technologies}: descreve as tecnologias, legíveis por máquina, que podem ser utilizadas por este \textit{standard}, incluindo a estrutura de dados lógica e o mapeamento de dados por cada tecnologia;
	\item \textbf{Parte 3} -- \texttt{Access Control, Authentication and Integrity Validation}: descreve as capacidades de segurança eletrónica que podem incorporar este \textit{standard}, incluindo mecanismos para controlo de acesso aos dados, verificação da origem de uma IDL e confirmação da integridade dos dados;
	\item \textbf{Parte 4} -- \texttt{Test Methods}: descreve métodos de teste que podem ser utilizados para determinar se uma IDL está de acordo com os requisitos das tecnologias legíveis por máquinas especificadas na parte 2 e com as capacidades de segurança eletrónica especificadas na parte 3.
\end{itemize}

Este \textit{standard} cria uma base comum para a utilização internacional e reconhecimento mútuo da IDL, sem impedir que países ou estados apliquem as suas regras de privacidade e que autoridades nacionais/comunitárias/regionais de trânsito tratem das suas necessidades específicas.

Assim, estas partes do ISO/IEC 18013 permitem estandardizar uma carta de condução eletrónica, com um cartão *chip* com ou sem contacto. É utilizada tecnologia derivada do passaporte eletrónico e exemplos de mecanismos de segurança utilizados para controlo de acesso são BAP (\textit{Broadcast Authentication Using Cryptographic Puzzles}), SAC/PACE (\textit{Supplemental Access Control} / \textit{Password Authenticated Connection Establishment}) e EAP (\textit{Extensible Authentication Protocol}).

A \textbf{Parte 5} do ISO/IEC 18013 -- \texttt{Mobile Driving Licence} -- pretende estabelecer um \textit{standard} para especificações de interface para a implementação de cartas de condução associadas a dispositivos móveis (\textit{Mobile Driving License} - mDL), isto é, dispositivos eletrónicos com interface de utilizador, a capacidade de armazenar informação da mDL e de a partilhar com um leitor, após instrução do titular (\textit{smartphones}, \textit{wearables}, entre outros). Um leitor mDL é um dispositivo portátil ou computador, que pode trocar dados com uma mDL, enquanto que o titular da mDL é o indivíduo para quem a mDL é emitida, isto é, o titular legítimo dos privilégios de condução refletidos na mDL.

Esta parte do ISO/IEC 18013 descreve a interface e requisitos físicos e funcionais associados que possibilitam a utilização de dispositivos móveis pelo titular da carta de condução, para a fornecer a um verificador, facilitando o acesso do mesmo a informação da carta de condução.

O objetivo é permitir que verificadores não associados à autoridade de emissão da mDL, como outras autoridades de emissão ou entidades verificadoras de outros países, ganhem acesso à informação para a qual o titular da mDL providenciar consentimento, conseguindo autenticá-la. Para o conjunto de informações disponibilizado pelo titular da mDL, estas entidades deverão poder:

\begin{enumerate}
	\item Utilizar uma máquina para obter a informação da mDL;
	\item Estabelecer a conexão entre a mDL e o seu titular com um grau aceitável de confiança;
	\item Autenticar a origem da informação mDL;
	\item Verificar a integridade da informação mDL.
\end{enumerate}

Outra vantagem das mDL em relação às cartas de condução físicas é a capacidade de atualizar informação com mais frequência e autenticá-la com um nível de confiança superior.

Existem três interfaces fulcrais para esta parte do *standard*, que são explicadas de seguida:

\begin{enumerate}
	\item Interface entre a mDL e a autoridade emissora, que permite controlar, entre outros, como o mDL é fornecido e como são efetuadas atualizações. Esta interface não é o foco desta parte do ISO/IEC 18013, uma vez que a interoperabilidade entre autoridades emissoras não é requerida para as funcionalidades pretendidas.
	\item Interface entre a mDL e o leitor, que tem de funcionar em tempo real e é descrita nesta parte do ISO/IEC 18013.
	\item Interface entre a autoridade emissora e a entidade de verificação, que facilita a troca de informação requerida para permitir a um leitor confirmar a autenticidade da informação da mDL e, em alguns casos, ler alguma informação da mesma. Esta interface é estabelecida preferencialmente entre a entidade verificadora e a autoridade emissora (diretamente ou através de intermediários), em vez de diretamente entre o leitor e a autoridade de emissão. Para além disso, não precisa de funcionar em tempo real e pode ser usada pela própria autoridade emissora, em leitores sob o seu controlo. Esta interface será descrita nesta parte do ISO/IEC 18013.
\end{enumerate}

\begin{figure}[H]
	\centering
	\includegraphics[width=0.8\textwidth]{images/interfaces.png}
	\caption{Ecossistema mDL, incluindo interfaces associadas}
	\label{fig:interfaces}
\end{figure}

% TODO: MAFALDA - REVER e RESUMIR

% O que cada parte trata, de forma geral.
% O qua a parte 5 trata de forma específica (o que é, objetivos, interfaces que trata, ...)
